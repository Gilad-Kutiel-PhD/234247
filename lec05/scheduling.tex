נתונות $n$ משימות 
$A = \{a_1, \ldots, a_n\}$
נסמן ב-%
$t(a_i)$
את הזמן הנדרש לביצוע משימה 
$a_i$
וב-%
$d(a_i)$
את זמן הסיום הרצוי של המשימה.
בהינתן סדר ביצוע המשימות (פרמוטציה)
$\pi:A \to [n]$.
נסמן ב-%
$\delta(a_i)$
את זמן הסיום של המשימה 
$a_i$,
כלומר
$$\delta(a_i) = \sum_{i \leq \pi(a_i)} t(\pi^{-1}(i))$$
נסמן ב-%
$l(a_i) \defeq \delta(a_i) - d(a_i)$
את האיחור בביצוע משימה 
$a_i$.
רוצים למצוא סדר שממזער את האיחור המקסימלי, כלומר
$$
\text{arg}\min_\pi \{\max_i l(a_i)\}
$$

האלגוריתם החמדן יבצע את המשימות בסדר לא יורד של זמני הסיום הרצויים.

\textbf{הוכחת נכונות}

נוכיח באינדוקציה את הטענה הבאה:

לכל $i$ קיים פתרון אופטימלי שמבצע את $i$ המשימות הראשונות לפי זמני הסיום שלהן.

בסיס: טריוויאלי

צעד: נסתכל על המשימה, $a$, שזמן הסיום שלה הוא ה-%
$i+1$
לפי סדר לא יורד. 
אם הפתרון האופטימלי מבצע את המשימה הזאת בזמן 
$i + 1$
סיימנו, אחרת הוא מבצע אותה בזמן 
$j > i + 1$
נסתכל על סדר ביצוע המשימות מזמן 
$i + 1$
עד זמן 
$j$:
$$
b_{i + 1}, \ldots, b_{j - 1}, a
$$

נבחן פתרון שמבצע את המשימות הללו בסדר הבא:

$$
a, b_{i + 1}, \ldots, b_{j - 1}
$$

נבדוק את האיחור המקסימלי של משימות אלו (האיחור המקסימלי של יתר המשימות לא השתנה) 
ונניח בשלילה שהוא גדל (אחרת סיימנו). 
אם זמן הסיום גדל זה חייב להיות בגלל אחת מהמשימות 
$b_{i+1}, \ldots, b_{j - 1}$,
נסמן אותה ב-$b$. נסמן את זמן הסיום שלה לפי הסדר החדש ב-%
$\delta'(b)$
אנחנו יודעים אבל ש-%
$\delta'(b) \leq \delta(a)$
וגם ש-%
$d(a) \leq d(b)$
ולכן
$\delta'(b) - d(b) \leq \delta(a) - d(a)$.

