נתונים $n$ אינטרוולים
$A = (a_1,\ldots,a_n)$
לכל אינטרוול זמן התחלה 
$s(a_i)$,
זמן סיום
$e(a_i) > s(a_i)$
ומשקל
$w(a_i)$.
תת קבוצה 
$I \subseteq A$
של אינטרוולים  נקראת בלתי תלויה אם לכל 
$a_i, a_j \in I$
אחד מהשניים מתקיים:
\begin{enumerate}
\item
$s(a_j) > e(a_i)$
\item
$s(a_i) > e(a_j)$
\end{enumerate}

רוצים למצוא קבוצה בלתי תלויה של אינטרוולים עם משקל מקסימלי.

\textbf{דוגמה:}
קלט לבעיה וקבוצה בלתי תלויה במשקל 13.
\begin{center}
\begin{tikzpicture}[very thick]
\draw[very thin, gray!30, dashed] (0,0) grid (10, 3);
\foreach \y \a \b \w in {
	0/0/1/2,0/2/4/3,0/5/8/4,0/9/10/2%	
	,1/0/2/3,1/3/6/2,1/7/10/4%	
	,2/1/3/5,2/4/5/4,2/6/9/3%	
	,3/0/4/2,3/5/7/3,3/8/10/4%	
}{
	\draw (\a,\y) -- (\b,\y) node[label above]{\w};
}

\begin{scope}[yshift=-4cm, gray!90]
\draw[very thin, gray!30, dashed] (0,0) grid (10, 3);
\foreach \y \a \b \w in {
	0/0/1/2,0/2/4/3,0/5/8/4,0/9/10/2%	
	,1/0/2/3,1/3/6/2,1/7/10/4%	
	,2/1/3/5,2/4/5/4,2/6/9/3%	
	,3/0/4/2,3/5/7/3,3/8/10/4%	
}{
	\draw (\a,\y) -- (\b,\y) node[label above]{\w};
}
\foreach \y \a \b \w in {
	,1/7/10/4%	
	,2/1/3/5,2/4/5/4%	
}{
	\draw[blue] (\a,\y) -- (\b,\y) node[label above, blue]{\w};
}
\end{scope}

\end{tikzpicture}
\end{center}


\textbf{אלגוריתם:}
נסתכל על קבוצת האינטרוולים ממוינת בסדר לא יורד של זמני הסיום
$A = (a_1,\ldots,a_n)$
ונסמן 
$A_i = (a_1,\ldots,a_i)$.
נגדיר
$$p(i) = \max
\begin{cases}
\max \{j : e(a_j) < s(a_i)\}
\\
0
\end{cases}
$$
כלומר 
$p(i)$
הוא האינדקס המקסימלי של אינטרוול
$a_j$
שמסתיים לפני שהאינטרוול 
$a_i$
מתחיל או 0 אם לא קיים כזה.

נגדיר את 
$O(i)$
להיות המשקל של תת קבוצה בלתי תלויה של 
$A_i$
עם משקל מקסימלי, אז
$O(n)$
הוא הערך אותו אנחנו מחפשים.

\begin{claim}
$$
O(i) = \max
\begin{cases}
w(i) + O(p(i))
\\
O(i - 1)
\end{cases}
$$
כמו כן מתקיים ש-%
$O(0) = 0$.
\end{claim}

\begin{proof}
באינדוקציה על $i$.

בסיס: עבור 
$i = 0$
טריוויאלי.

עבור 
$i + 1$
נקבע פתרון אופטימלי $OPT$ ונסמן
$OPT(i) = OPT \cap A_i$. 

אם
$a_{i+1} \in OPT$
אז $OPT$ לא יכול להכיל אף אינטרוול
$a_{p(i+1)}, \ldots, a_{i+1}}$
לפי הנחת האינדוקציה
$$O(p(i + 1)) \geq OPT(p(i + 1)$$
ולכן הטענה מתקיימת כי
$$O(p(i + 1)) + w(a_{i+1}) \geq OPT(p(i + 1) + w(a_{i+1})$$

מצד שני, אם 
$a_{i+1} \notin OPT$
אז לפי ההנחה 
$$O(i) \geq OPT(i)$$
והטענה מתקיימת.
\end{proof}

\subsection*{חישוב יעיל של $O$}
כיצד נחשב את $O$ ביעילות ?
נשים לב שאם מחשבים את ערכי $O$ מ-1 עד $n$ ושומרים את הערכים (למשל במערך) אז חישוב של 
כל ערך לוקח 
$O(1)$
זמן.

זמן הריצה של האלגוריתם:
\begin{enumerate}
\item
מיון - 
$O(n\log n)$
\item
חישוב $p$ - 
$O(n\log n)$
(חיפוש בינארי לכל $i$)
\item
חישוב $O$ -
$O(n)$
\end{enumerate}

סך הכל 
$O(n\log n)$

\textbf{דוגמת הרצה:}
\begin{center}
\begin{tikzpicture}[very thick]
\draw[very thin, gray!30, dashed] (0,0) grid (10, 2);
\foreach \y \a \b \w in {
	0/0/1/2,0/2/4/3,0/5/8/4,0/9/10/2%	
	,1/0/2/3,1/3/6/2,1/7/10/4%	
	,2/1/3/5,2/4/5/4,2/6/9/3%	
}{
	\draw (\a,\y) -- (\b,\y) node[label above]{\w};
}
\end{tikzpicture}
\end{center}

