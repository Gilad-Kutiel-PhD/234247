האלגוריתם הגנרי של פורד פלקרסון:
\begin{enumerate}
\item
אתחול: מציבים 
$f(e) \leftarrow 0$
לכל
$e \in E$

\item
כל עוד יש מסלול שיפור ברשת השיורית
$(G_f, s, t, c_f)$
\begin{enumerate}
\item
מציבים ב-$f$ את הזרימה המשופרת לפי למת שיפור הזרימה
\end{enumerate}
\item
פולטים את $f$
\end{enumerate}

ראינו שאלגוריתם זה אינו פולינומי אפילו כאשר כל הקיבולים שלמים (ואף במקרה הכללי הוא אינו עוצר כלל).

מקרה פרטי של אלגוריתם זה הוא האלגוריתם של אדמונדס וקרפ:

\begin{enumerate}
\item
אתחול: מציבים 
$f(e) \leftarrow 0$
לכל
$e \in E$

\item
כל עוד יש מסלול שיפור ברשת השיורית
$(G_f, s, t, c_f)$
\begin{enumerate}
\item
יהי $P$ מסלול קצר ביותר מ-$s$ ל-$t$.
\item
שפר לפי $P$ והצב ב-$f$ את הזרימה המשופרת לפי למת שיפור הזרימה
\end{enumerate}
\item
פולטים את $f$
\end{enumerate}


כעת נראה שאלגוריתם זה הוא פולינומי ללא תלות בפונקציית הקיבול.

נסמן ב-%
$f_1, f_2 \ldots $
את פונקציית הזרימה שמחשב האלגוריתם בכל איטרציה, וב-%
$d_{f_i}(v)$
את המרחק של הצומת $v$ מהצומת $s$ ברשת השיורית.

\begin{claim}
לכל $i$ ולכל $v$ מתקיים ש-%
$d_{f_i}(v) \leq d_{f_{i+1}}(v)$.
\end{claim}

\begin{proof}
עבור $i$ נתון, נוכיח באינדוקציה על $k$ - המרחק של $v$ מ-$s$ ב-%
$G_{f_{i + 1}}$.

\textbf{בסיס:}
עבור 
$k=0$
טריוויאלי.

\textbf{צעד:}
עבור צומת $v$ במרחק 
$k + 1$ 
מ-$s$ ומסלול
$s = v_0, \ldots, v_k, v_{k + 1} = v$
מתקיים (לפי הנחת האינדוקציה) ש:
$$d_{f_i}(v_k) \leq d_{f_{i + 1}}(v_k)$$
אם הקשת 
$v_k v_{k + 1}$
קיימת ב-%
$G_{f_i}$
אז סיימנו.

אחרת במסלול השיפור ב-%
$G_{f_i}$
קיימת הקשת 
$v_{k+1}v_k$
ומכאן:
$$
d_{f_i}(v_{k + 1}) =
d_{f_i}(v_{k}) - 1 \leq 
d_{f_{i + 1}}(v_{k}) - 1 =
d_{f_{i + 1}}(v_{k + 1}) - 2
$$
\end{proof}

\begin{corollary}
לכל $v$ ולכל
$i < j$
מתקיים
$f_i(v) \leq f_j(v)$

\end{corollary}

\begin{corollary}
אם
$uv \in E_{f_{i + 1}}$
ו-%
$uv \notin E_{f_{i}}$
אז
$d_{f_{i + 1}}(v) \geq d(f_{i}(v)) + 2$.
\end{corollary}
