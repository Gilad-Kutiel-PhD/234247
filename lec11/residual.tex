\textbf{הנחה:}
נניח בלי הגבלת הכלליות (למה?) שברשת הזרימה (המקורית) אין קשתות אנטי-מקביליות.

\begin{definition}[רשת שיורית]
בהינתן רשת זרימה,
$N = (G, s, t, c)$
וזרימה, $f$, הרשת השיורית היא
$N_f = (G_f, s, t, c_f)$
כאשר אם 
$G = (V, E)$ 
אז

$$
\begin{array}{l}
G_f = (V, E_f = E \cup \overline{E})
\\
\overline{E} = \{\overline{e} = vu : e = uv \in E\}
\\
c_f(\overline{e}) \defeq f(e)
\\
c_f(e) \defeq c(e) - f(e)
\\
\end{array}
$$
\end{definition}

\textbf{דוגמה:}

\begin{center}
\begin{tikzpicture}[
	x=2cm
	,y=2cm
]

\node(s) at(0,0) {s};
\foreach[count=\i] \x \y in {
	1/1,1/-1
}{
	\node(\i) at(\x,\y){\i};
}
\node(t) at(2,0) {t};

\foreach \u \v \c \f in{
	s/1/5/1, s/2/3/1%
	,1/t/3/1%
	,2/t/1/1%
}{
	\draw (\u) -- node[label inside] {\f/\c} (\v);
}

\begin{scope}[xshift=7cm]
\node(s) at(0,0) {s};
\foreach[count=\i] \x \y in {
	1/1,1/-1
}{
	\node(\i) at(\x,\y){\i};
}
\node(t) at(2,0) {t};

\foreach \u \v \c in{
	s/1/4, s/2/2%
	,1/s/3, 1/t/2%
	,2/s/1,2/t/1%
	,t/1/1, t/2/0%
}{
	\draw (\u) to[bend left] node[label inside] {\c} (\v);
}
\end{scope}
\end{tikzpicture}
\end{center}

\begin{definition}[חיבור זרימות]
אם $f$ זרימה ב-
$G$
ו-$g$ זרימה ב-
$G_f$
נגדיר את הסכום שלהן להיות:
$$
\forall e \in E: \; h(e) = f(e) + g(e) - g(\overline{e})
$$
\end{definition}

\begin{lemma}
\label{lemma:addition}
$h$
היא פונקציית זרימה ב-$N$
ומתקיים 
$|h| = |f| + |g|$.
\end{lemma}

\begin{proof}
$ $\\
\textbf{חוק הקשת:}

$
c(e) \geq
f(e) + c(e) - f(e) - g(\overline{e}) \geq
f(e) + g(e) - g(\overline{e}) \geq 
f(e) + g(e) - f(e) \geq 0
$



\textbf{חוק הצומת:}
$$
\begin{array}{*2{>{\displaystyle}l}r}
\sum_{uv \in E} h(uv) - \sum_{vw \in E} h(vw) = 
\\
&	\sum_{uv \in E} f(uv) + g(uv) - g(vu) - 
	\sum_{vw \in E} f(vw) + g(vw) - g(wv) = 
\\
&	\sum_{uv \in E} f(uv) -
	\sum_{vw \in E} f(vw) +
\\
&	\sum_{uv \in E} (g(uv) - g(vu)) -
	\sum_{vw \in E} (g(vw) - g(wv)) = & 	0 + 0 + 0
\end{array}
$$
\end{proof}

\begin{lemma}
\label{lemma:path}
אם $P$ מסלול (פשוט) מ-$s$ ל-$t$ ברשת זרימה 
$(G, s, t, c)$
ו-%
$\varepsilon$
הקיבול המינימלי של קשת ב-$P$ אז הפונקציה:
$$
f_P(e) = 
\begin{cases}
\varepsilon & \text{if} \; e \in P
\\
0 & \text{otherwise}
\end{cases}
$$
היא פונקציית זרימה.
\end{lemma}

\begin{proof}
חוק הקשת מתקיים לפי ההגדרה של $f$ ושל 
$\varepsilon$.
עבור צומת פנימי במסלול, $v$, מתקיים ש-%
$f(v) = f(vw) - f(uv) = 0$
כאשר $u$ ו-$w$ הצמתים לפני ואחרי $v$ במסלול בהתאמה.
\end{proof}

\begin{definition}[מסלול שיפור]
בהינתן רשת זרימה,
$(G, s, t, c)$,
וזרימה, $f$, 
מסלול שיפור הוא מסלול (פשוט) מ-$s$ ל-$t$ ב-%
$G_f$.
\end{definition}

\begin{lemma}
\label{lemma:improve}
אם $P$ הוא מסלול שיפור ביחס לרשת זרימה 
$(G, s, t, c)$
וזרימה
$f$
אז
$h = f + f_P$
זרימה חוקית ו-%
${|h| = |f| + \varepsilon}$.
\end{lemma}

\begin{proof}
נובע ישירות מלמות
\ref{lemma:addition}
ו-%
\ref{lemma:path}.
\end{proof}
