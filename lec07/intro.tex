נתון לנו גרף (מכוון או לא) 
$G = (V, E)$
וכן פונקציית משקל על הקשתות 
$w:E \to \mathbb{R}$.
נסמן ב-%
$P_{st} = (s = v_0, \ldots, v_k = t)$
מסלול מצומת $s$ לצומת $t$
וב-%
$\delta(s,t)$
את משקל המסלול הקל ביותר בין שני צמתים $s$ ו-$t$.
כלומר:
$$
\delta(s,t) = \inf_{P_{st}} w(P_{st})
$$

\textbf{דוגמה:}
למה שווה 
$\delta(1,3)$
בגרף הבא ?
למה שווה
$\delta(1,7)$
?

\begin{center}
\begin{tikzpicture}[every node/.style={default node}]
\foreach[count=\i] \x \y in {
	0/0,1/2,2/1,3/-1,4/2,5/0,6/1
}{
	\node(\i) at(\x,\y){\i};
}

\foreach \u \v \w in {
	1/2/2,1/3/2,1/4/2%
	,2/3/3,2/5/3%
	,3/4/3-,3/5/3-,3/6/3-%
	,4/6/6%
	,5/6/6,5/7/5%
	,6/7/5%
}{
	\draw (\u) -- (\v) node[label above]{\w};
}

\end{tikzpicture}
\end{center}

\textbf{הערות:}
\begin{itemize}
\item
לאלגוריתמים למציאת מסלול קל ביותר שימושים רבים, 
אולי המידי שבהם הוא חישוב מסלול קצר ביותר בין שתי נקודות במפה.
\item
יתכנו משקלים שלילים על הקשתות, 
למשל אם אנחנו מעוניינים לתכנן מסלול לרכב חשמלי והמטרה שלנו היא לחסוך בסוללה.
\item
כאשר צומת $t$ לא ישיג מצומת $s$ נגדיר 
$\delta(s,t) = \infty$
\item
כאשר יש מעגל שלילי ישיג מצומת $s$, נגדיר
$\delta(s,v) = -\infty$ 
לכל $v$ שישיג מ-$s$ (בדרך כלל במקרה כזה רק נרצה לזהות שזהו אכן המצב).
\end{itemize}
