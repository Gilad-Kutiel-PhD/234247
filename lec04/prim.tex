אלגוריתם פרים מתחיל מעץ שמכיל צומת אחד ובכל איטרציה מוסיף לעץ את הקשת הכי זולה שקצה
אחד שלה בדיוק נוגע בעץ.
דוגמה:

\begin{center}
\begin{tikzpicture}[
every node/.style={default node}
,x=1.7cm
,y=1.7cm
]

\foreach[count=\i] \x \y in {
0/0,0/1,1/0,1/1,-1/-1,-1/2,2/-1,2/2}{
\node(\i) at (\x, \y) {\i};
}

\foreach \u \v \w in {
3/1/1%
,1/2/2%
,2/4/2%
,4/3/2%
,5/6/3%
,6/8/4%
,8/7/5%
,7/5/6%
,6/2/7%
,8/4/8%
,7/3/8%
,5/1/8%
}{
\draw (\u) -- (\v) node[label above]{\w};
}

\begin{scope}[xshift=8cm]
\foreach[count=\i] \x \y in {
0/0,0/1,1/0,1/1,-1/-1,-1/2,2/-1,2/2}{
\node(\i) at (\x, \y) {\i};
}

\foreach[count=\i] \u \v in {
5/6%
,6/8%
,6/2%
,2/4%
,2/1%
,1/3%
,8/7%
}{
\draw (\u) -- (\v) node[label below, blue]{\i};
}
\end{scope}

\end{tikzpicture}
\end{center}

\newpage
פורמלית:
\begin{enumerate}
\item
אתחול:
$U \leftarrow \{u\}, B \leftarrow \emptyset$, 
כאשר $u$ צומת שרירותי.
\item
כל עוד 
$U \neq V$
\begin{enumerate}
הפעל את הכלל הכחול על החתך $U$ ועדכן את $B$
\end{enumerate}
\begin{observation}
אלגוריתם פרים הוא מימוש של האלגוריתם הכללי שמפעיל את הכלל הכחול.
\end{observation}
\end{enumerate}
