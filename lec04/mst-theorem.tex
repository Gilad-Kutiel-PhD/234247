\begin{definition}
בהינתן עץ 
$T = (V, F)$
וקשת 
$e \notin F$
נסמן ב-%
$C_e$
את המעגל שמכיל את $e$ בגרף 
$(V, F \cup \{e\})$
\end{definition}
\begin{definition}
בהינתן עץ 
$T = (V, F)$
וקשת 
$e = uv \in F$
נסמן ב-%
$U_e$
את החתך שמכיל את כל הצמתים שישיגים מ-$u$ בגרף
$(V, F \setminus \{e\})$
\end{definition}

\begin{theorem}
עץ פורש 
$T = (V, F)$
של גרף 
$G = (V, E)$
הוא מינימלי אמ"מ לכל 
$e \in F$
מתקיים ש-$e$ קשת במשקל מינימלי שחוצה את 
$U_e$.
באופן שקול, $T$ מינימלי אמ"מ לכל 
$e \in E \setminus F$
מתקיים ש-$e$ קשת במשקל מקסימלי במעגל 
$C_e$
\end{theorem}
\begin{proof}
נוכיח כיוון אחד. 
נניח שמתקיים שלכל 
$e \in F$
מתקיים ש-$e$ קשת במשקל מינימלי שחוצה את 
$U_e$.
אז עץ כזה מתקבל על ידי הפעלה של הכלל הכחול על אוסף החתכים 
$\{U_e : e \in F\}$.
באופן דומה, נניח שלכל 
$e \in E \setminus F$
מתקיים ש-$e$ קשת במשקל מקסימלי במעגל 
$C_e$
אז עץ כזה מתקבל על ידי הפעלת הכלל האדום על אוסף המעגלים 
$\{C_e : e \in E \setminus F\}$
\end{proof}
