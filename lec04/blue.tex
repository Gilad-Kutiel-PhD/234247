בחלק זה נראה אלגוריתם גנרי שפועל על פי הכלל הכחול.

\begin{definition}[קשת קלה]
בהינתן חתך, $S$, ופונקצית משקל, 
$w: E \to \mathbb{R}$,
קשת $uv$ שחוצה את $S$ תקרא קלה אם לא קיימת קשת אחרת שחוצה את $S$, 
$u'v'$
שמקיימת 
$w(u'v') < w(uv)$.
\end{definition}
\begin{enumerate}
\item
אתחול:
$F \leftarrow \emptyset$ 
(קשתות כחולות)
\item
כל עוד 
$T = (V, F)$
אינו קשיר
\begin{enumerate}
\item
בחר חתך לבן, $S$, וקשת קלה, $uv$, שחוצה אותו 
\item
$F \leftarrow F \cup \{uv\}$
\end{enumerate}
\end{enumerate}

\begin{claim}
האלגוריתם מחזיר עץ
\end{claim}
\begin{proof}
קשירות נובעת מיידית מהגדרת האלגוריתם.
נניח בשלילה שנסגר מעגל.
נסתכל על הנקודה בה האלגוריתם צובע את הקשת שסוגרת את המעגל הכחול.
לפי אבחנה 
\ref{observation:cycle}
קיימת קשת כחולה נוספת שחוצה את החתך - סתירה.
\end{proof}

\begin{claim}
האלגוריתם מחזיר עץ פורש מינימלי
\end{claim}

\begin{proof}
נוכיח באינדוקציה שבכל שלב בריצת האלגוריתם אוסף הקשתות הכחולות מוכל בתוך עץ פורש מינימלי כלשהו:

בסיס: טריוויאלי באתחול

צעד: לפי ההנחה קיים עפ"מ שמכיל את $i$ הקשתות הכחולות הראשונות.
נניח שהקשת, $e$, שהוספנו בשלב ה-%
$i+1$
לא שייכת לעפ"מ הנ"ל (אחרת סיימנו).
נוסיף את $e$ לעפ"מ, סגרנו מעגל.
במעגל זה קיימת קשת,
$e' \neq e$,
שחוצה את $S$, החתך (הלבן) שגרם להוספת $e$.
לפי הגדרת האלגוריתם
$w(e) \leq w(e')$
ולכן ניתן להחליף בין הקשתות הנ"ל ולקבל עפ"מ שמכיל גם את $e$.
\end{proof}


