מעתה נעסוק רק בגרפים לא מכוונים וקשירים (בלי הגבלת הכלליות).
\begin{definition}[צומת הפרדה]
צומת $v$ יקרא 
\underline{צומת הפרדה}
אם
$G[V \setminus \{v\}]$
אינו קשיר
\end{definition}
\textbf{דוגמה:}
צומת 3 בגרף הבא הוא צומת הפרדה (ורק הוא)
\begin{center}
\begin{tikzpicture}[every node/.style={default node}]
\foreach[count=\i] \x \y in {0/0, -1/-1, 1/-1, 0/-2, 2/-2}{
	\node(\i) at(\x, \y) {\i};
}

\foreach \u \v in {%
	1/2%
	,1/3%
	,2/3%
	,3/4%
	,3/5%
}{
	\draw (\u) -- (\v);
}
\end{tikzpicture}
\end{center}

מה המשמעות של צמתי הפרדה ? ברשת כבישים ? רשת תקשורת ? רשת חברתית ?

אלגוריתם טריוויאלי למציאת צמתי הפרדה:
\begin{enumerate}
\item
עבור כל צומת $v$
	\begin{enumerate}
	\item
	מחק את $v$ מ-$G$
	\item
	בדוק אם $G$ קשיר
	\end{enumerate}
\end{enumerate}

מה הסיבוכיות ? נרצה לעשות יותר טוב.

\textbf{שאלה:}
מי הם צמתי ההפרדה בעצים?
\begin{center}
\begin{tikzpicture}[every node/.style={default node}]
\foreach[count=\i] \x \y in {0/0, -1/-1, 1/-1, 0/-2, 2/-2}{
	\node(\i) at(\x, \y) {\i};
}

\foreach \u \v in {%
	1/2%
	,1/3%
	,3/4%
	,3/5%
}{
	\draw (\u) -- (\v);
}
\end{tikzpicture}
\end{center}
נסתכל על גרף לא מכוון כאיחוד של עץ DFS וקבוצת קשתות אחוריות.
מה יכול לקרות כאשר מוציאים קשת שאינו עלה מהגרף ?

למשל מה יקרה אם נסיר את צומת 5 מהגרף הבא ?
\begin{center}
\begin{tikzpicture}[every node/.style={default node}]
\foreach[count=\i] \x \y in {0/0, -1/-1, 1/-1, 0/-2, 2/-2}{
	\node(\i) at(\x, \y) {\i};
}

\foreach \u \v in {%
	1/2%
	,1/3%
	,3/4%
	,3/5%
}{
	\draw (\u) -- (\v);
}

\begin{scope}[
every node/.style={
	regular polygon
	,regular polygon sides=3
	,minimum height=2cm
	,draw
}]
\node(6) at(1, -4) {6};
\node(7) at(3, -4) {7};
\end{scope}

\draw (5) -- (6.north);
\draw (5) -- (7.north);
\draw[dashed] (7.north east) to[out=45, in=0] (3);
\end{tikzpicture}
\end{center}

קל להשתכנע שתת העץ 7 ישאר מחובר לגרף בעוד שתת העץ 6 יתנתק מהגרף.

בעץ מושרש, $T$, נסמן ב-%
$T_v$
את תת העץ ששורשו הוא $v$. כלומר תת העץ שמכיל את $v$ ואת כל צאצאיו.
\begin{definition}[קשת עוקפת]
נגיד שקשת בגרף מצאצא של $u$ לאב קדמון של $u$ (שניהם לא $u$ עצמו) עוקפת את $u$
\end{definition}

\begin{definition}[בן מפריד]
צומת $v$ עם אבא $u$ יקרא בן מפריד אם לא קיימת ב-%
$T_v$
קשת שעוקפת את $u$
\end{definition}
\begin{claim}
צומת $u$ בעץ DFS שאינו עלה הוא מפריד אם ורק אם  יש לו בן מפריד.
בפרט שורש העץ הוא צומת מפריד אמ"מ הוא אינו עלה (יש לו יותר מבן אחד)
\end{claim}
\begin{proof}
נזכיר שבגרף אין קשתות חוצות, כמו כן נניח ש-$u$ אינו השורש (ניתן להוכיח נכונות לגבי השורש בנפרד)

כיוון ראשון: נניח שמ-%
$T_v$
אין קשת לאב קדמון של $u$ אזי כל מסלול מהאבא של $u$ ל-%
$T_v$
חייב לעבור ב-$u$.

כיוון שני: נניח שלכל בן $v$ של $u$ קיימת קשת עוקפת מ-%
$T_v$,
$wx$
נשים לב שהוספת הקשת $wx$ סוגרת מעגל שמכיל את הקשת $uv$ ולכן אם נסיר את הקשת $uv$
נקבל שוב עץ, נחזור על הפעולה הזאת לכל בן של $u$ ולבסוף נסיר את $u$ מהעץ. 
קיבלנו שוב עץ ולכן $u$ אינו צומת הפרדה.

\end{proof}

\begin{definition}
נגדיר
$$
L(u) \defeq \min_{vw \in E: v \in T_u} \alpha(w)
$$
\end{definition}
כלומר הצומת עם ערך $\alpha$ מינימלי שהוא שכן של 
$T_u$.

\textbf{דוגמה:}
מה ערכי $L$ בגרף הבא ?
\begin{center}
\begin{tikzpicture}[every node/.style={default node}]
\foreach[count=\i] \x \y in {
0/0
	,-1/-1
	,1/-1
		,2/-2
		,0/-2
			,-1/-3
			,1/-3
				,0/-4
				,2/-4
}{
	\node(\i) at(\x, \y) {\i};
}

\foreach \u \v in {%
	1/2%
	,1/3%
	,3/4%
	,3/5%
	,5/6%	
	,5/7%
	,7/8%	
	,7/9%				
}{
	\draw (\u) -- (\v);
}

\foreach \u \v in {%
	8/5%
	,5/1%	
}{
	\draw[dashed] (\u) -- (\v);
}

\draw[dashed] (9) to[out=45, in=0] (3);

\end{tikzpicture}
\end{center}

נסמן ב-%
$C(u)$
את קבוצת הבנים של $u$ ב-$T$.
\begin{observation}
צומת $u$ הוא צומת מפריד אמ"מ 
$\displaystyle\max_{v \in C(u)} L(v) \ge \alpha(u)$
\end{observation}

כלומר, כדי למצוא אלגוריתמית את צמתי ההפרדה כל שעלינו לעשות הוא לחשב את ערכי $L$.
נשים לב  שמתקיימת הנוסחה (הרקורסיבית) הבאה:
$$
L(u) = \min
\begin{cases}
\displaystyle\min_{uv \in E} \alpha(v)
\\
\displaystyle\min_{v \in C(u)} L(v)
\end{cases}
$$




\newpage
נעדכן את המימוש של DFS כך שבריצת האלגוריתם נחשב גם את ערכי $L$
\begin{enumerate}
\item
אתחול:
$\ldots$
\colorbox{yellow}{
לכל 
$v \in V$
מציבים
$L(v) \leftarrow \infty$
}
\item
כל עוד המחסנית לא ריקה
\begin{enumerate}
	\item
	$u \leftarrow S.top()$
	\item 
	אם קיימת קשת 
	$uv$
	שחוצה את $U$ 
	($u \in U$)
		\begin{enumerate}
		\item
		$U \leftarrow U \cup \{v\}, F \leftarrow F \cup \{uv\}$
		\item
		$p(v) \leftarrow u$
		\item
		$\alpha(v) \leftarrow i$
		\item
		$S.push(v)$
		\end{enumerate}
	\item
	אחרת 
	\begin{enumerate}
		\item 
		\colorbox{yellow}{$L(u) \leftarrow \displaystyle\min_{uv \in E} \alpha(v)$}
		\item \colorbox{yellow}{$L(p(u)) \leftarrow \min\{L(u), L(p(u))\}$}
		\item $u \leftarrow S.pop()$
		\item $\beta(u) \leftarrow i$
	\end{enumerate}
	\item
	$i \leftarrow i + 1$
	\end{enumerate}
\end{enumerate}

