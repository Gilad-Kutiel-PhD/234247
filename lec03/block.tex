\begin{definition}[גרף אי פריק]
גרף (קשיר) יקרא אי פריק אם אין בו צמתי הפרדה
\end{definition}

במילים אחרות, הגרף נשאר קשיר גם אחרי הסרת צומת כלשהו ממנו.

\begin{definition}[רכיב אי פריק]
רכיב אי פריק $H$ של $G$ הוא תת גרף (קשיר) אי פריק מקסימלי של $G$
\end{definition}

\textbf{דוגמה:}
\begin{center}
\begin{tikzpicture}[every node/.style={default node}]
\foreach[count=\i] \x \y in{0/0, 1/1, 1/-1, 2/0, 3/0, 4/1, 4/-1, 5/1}{
	\node(\i) at(\x, \y) {\i};
}
\foreach \u \v in {%
	1/2%
	,1/3%
	,2/4%	
	,3/4%	
	,4/5%	
	,5/6%	
	,5/7%	
	,6/7%	
	,6/8%	
}{
	\draw (\u) -- (\v);
}

\draw[dashed]
(1.west) to[bend left]
(2.north) to[bend left]
(4.east) to[bend left]
(3.south) to[bend left]
(1.west)
;

\draw[dashed]
(4.west) to[out=90, in=90]
(5.east) to[out=-90, in=-90]
(4.west) 
;

\draw[dashed]
(5.west) to[bend left]
(6.north) to[out=0,in=0]
(7.south) to[bend left]
(5.west) 
;

\draw[dashed]
(6.west) to[out=90, in=90]
(8.east) to[out=-90,in=-90]
(6.west) 
;

\end{tikzpicture}
\end{center}

\begin{claim}
\label{claim:disjoint}
לשני רכיבים אי פריקים
$H_1$
ו-%
$H_2$
צומת אחד משותף לכל היותר
\end{claim}
\begin{proof}
נניח בשלילה שישנם שני רכיבים כאלו שחולקים את הצמתים $u$ ו-$v$.
נשים לב שלאחר שהסרה של צומת כלשהו כל רכיב בנפרד נשאר קשיר.
מעבר לכך הרכיבים עדיין חולקים צומת משותף ולכן הרכיב שמתקבל מאיחוד שני הרכיבים גם הוא אי פריק.
סתירה למקסימליות.
\end{proof}

\begin{claim}
הרכיבים האי פריקים מהווים חלוקה של קשתות הגרף
\end{claim}
\begin{proof}
קשת היא גרף אי פריק ולכן מוכלת ברכיב אי פריק אחד לפחות. 
מטענה 
\ref{claim:disjoint}
נובע שגם לכל היותר

\end{proof}
\begin{claim}
כל מעגל ב-$G$ מוכל ברכיב פריק של $G$
\end{claim}
\begin{proof}
נובע ישירות מכך שמעגל הוא גרף אי פריק 
\end{proof}

נרצה למצוא את הרכיבים האי פריקים של גרף $G$.
\begin{claim}
עבור צומת הפרדה $u$ עם בן מפריד $v$, כל צמתי הרכיב האי פריק שמכיל את $uv$ (פרט ל-$u$)
נמצאים ב-%
$T_v$
\end{claim}
\begin{proof}
נשים לב ש-$u$ מפריד את 
$T_v$
משאר הגרף
\end{proof}
נסמן ב-$S$ את קבוצת הבנים המפרידים אז
\begin{corollary}
צמתי הרכיב האי פריק שמכיל את $uv$ הוא 
$T_v \cup \{v\} \setminus \bigcup_{w \in S:w \ne v} T_w$
\end{corollary}

%%%%%%%%%%%%%%%%%%%%%%%

נעדכן את המימוש של DFS כך שבריצת האלגוריתם נחשב גם את הרכיבים האי פריקים 
(נרצה לשמור לכל צומת את מספר הרכיב אליו הוא שייך)
\begin{enumerate}
\item
אתחול:
$\ldots$
\colorbox{yellow}{
$S' \leftarrow (s)$,
$b \leftarrow 0$,
לכל צומת 
$v \in V$
$B(v) \leftarrow -1$
}
\item
כל עוד המחסנית לא ריקה
\begin{enumerate}
	\item
	$u \leftarrow S.top()$
	\item 
	אם קיימת קשת 
	$uv$
	שחוצה את $U$ 
	($u \in U$)
		\begin{enumerate}
		\item $\ldots$
		\item
		\colorbox{yellow}{$S'.push(v)$}
		\end{enumerate}
	\item
	אחרת 
	\begin{enumerate}
		\item $v \leftarrow S.pop()$
		\item $\beta(v) \leftarrow i$

		\item 
				\colorbox{yellow}{
אם $v$ בן מפריד של $u$
		}
			\begin{enumerate}
				\item \colorbox{yellow}{$w \leftarrow S'.pop()$}
				\item \colorbox{yellow}{
כל עוד 
				$w \ne v$
				}
					\begin{itemize}
						\item \colorbox{yellow}{$B(w) = b$}
						\item \colorbox{yellow}{$w \leftarrow S'.pop()$}
					\end{itemize}	
			\item \colorbox{yellow}{$b \leftarrow b + 1$}
			\end{enumerate}			
	\end{enumerate}
	\item
	$i \leftarrow i + 1$
	\end{enumerate}
\end{enumerate}

\subsection*{עץ רכיבים אי פריקים}
עבור גרף $G$ נגדיר את גרף הרכיבים האי פריקים, 
$B(G)$,
כגרף הדו צדדי על קבוצות הצמתים $B$ ו-$S$ 
כאשר ב-$B$ צומת עבור כל רכיב אי פריק ב-$G$ וב-$S$ צומת עבור כל צומת הפרדה ב-$G$.
בגרף הנ"ל תהיה קשת $bs$, 
$b \in B$,
$s \in S$,
אמ"מ הרכיב שמתאים ל-$b$ מכיל את הצומת $s$.

נשים לב שכל מסלול ב-$G$ מתאים למסלול (יחיד) ב-%
$B(G)$
ולכן 
$B(G)$
קשיר.
כמו כן נשים לב שב-%
$B(G)$
לא קיימים מעגלים כי זה יגרור שקיים מעגל ב-$G$ שאינו שעובר ביותר מרכיב אי פריק אחד.
\begin{corollary}
$B(G)$
הוא עץ
\end{corollary}

\begin{center}
\begin{tikzpicture}[every node/.style={default node}, x=2cm]
\foreach[count=\i] \x \y in{0/0, 1/1, 1/-1, 2/0, 3/0, 4/1, 4/-1, 2/-2}{
	\node(\i) at(\x, \y) {\i};
}
\foreach \u \v in {%
	1/2%
	,1/3%
	,2/4%	
	,3/4%	
	,4/5%	
	,4/8%	
	,5/6%	
	,5/7%	
	,6/7%	
}{
	\draw (\u) -- (\v);
}

\draw[dashed]
(1.west) to[bend left]
(2.north) to[bend left]
(4.east) to[bend left]
(3.south) to[bend left]
(1.west)
;

\draw[dashed]
(4.west) to[out=90, in=90]
(5.east) to[out=-90, in=-90]
(4.west) 
;

\draw[dashed]
(5.west) to[bend left]
(6.north) to[out=0,in=0]
(7.south) to[bend left]
(5.west) 
;

\draw[dashed]
(4.north) to[out=0, in=0]
(8.south) to[out=180,in=-180]
(4.north) 
;

\begin{scope}[yshift=-4cm]
\foreach[count=\i] \x/\y/\n in{
	2/0/4
	,3/0/5
}{
	\node(s\i) at(\x, \y) {\n};
}
\foreach[count=\i] \x/\y/\n in{
	1/0/1234
	,2.5/0/45
	,2/-1/48
	,3.5/0/567
}{
	\node[rectangle](b\i) at(\x, \y) {\n};
}
\foreach \u \v in {%
	b1/s1%
	,s1/b2%
	,s1/b3%	
	,b2/s2%
	,s2/b4%
}{
	\draw (\u) -- (\v);
}
\end{scope}
\end{tikzpicture}
\end{center}

