בהינתן גרף מכוון, 
$G = (V, E)$
נגדיר את היחס הבא:
$\{(u,v) : u \rightsquigarrow v \land v \rightsquigarrow u\}$,
כלומר צמתים $u$ ו-$v$ ביחס אם קיים מסלול מ-$u$ ל-$v$ וקיים מסלול מ-$v$ ל-$u$.
נשים לב שזהו יחס שקילות ולכן הוא מגדיר מחלקות שקילות. 
למחלקות שקילות אלו נקרא קבוצת הרכיבים קשירים היטב (רק"ה) של $G$.

\textbf{דוגמה:}
\begin{center}
\begin{tikzpicture}[every node/.style={default node}, ->, x=1.5cm, y=1.5cm]

% NODES
\foreach[count=\i] \x \y in {
	0/0,0/1
	,1/0,1/1
	,2/0,2/1
	,3/0,3/1
}{
	\node(\i) at(\x,\y) {\i};
}

% EDGES
\foreach \u \v in{%
	1/2,1/3%
	,2/4%
	,4/1,4/3,4/6%
	,5/7%
	,6/5%
}{
	\draw[] (\u) -- (\v);
}

\foreach \u \v in{%
	3/5%
	,5/3%
	,6/8%
	,8/6%
}{
	\draw[] (\u) to[bend right] (\v);
}

\begin{scope}[xshift=7cm]
% NODES
\foreach[count=\i] \x \y in {
	0/0,0/1
	,1/0,1/1
	,2/0,2/1
	,3/0,3/1
}{
	\node(\i) at(\x,\y) {\i};
}

% EDGES
\foreach \u \v in{%
	1/2,1/3%
	,2/4%
	,4/1,4/3,4/6%
	,5/7%
	,6/5%
}{
	\draw[] (\u) -- (\v);
}

\foreach \u \v in{%
	3/5%
	,5/3%
	,6/8%
	,8/6%
}{
	\draw[] (\u) to[bend right] (\v);
}
\end{scope}

\begin{scope}[-, dashed, very thick]

\draw[orange!50]
(1.south west)	to[out=135, in=225]
(2.north west)	to[out=45, in=135]
(4.north east)	to[out=-45,in=-45,looseness=.7]
(1.south west)
;

\draw[blue!50]
(3.west)	to[out=90, in=90]
(5.east)	to[out=270, in=270]
(3.west)
;

\draw[green!50]
(6.west)	to[out=90, in=90]
(8.east)	to[out=270, in=270]
(6.west)
;

\draw[red!50, looseness=2]
(7.west)	to[out=90, in=90]
(7.east)	to[out=270, in=270]
(7.west)
;

\end{scope}

\end{tikzpicture}
\end{center}

נסמן את הרק"ה של גרף ב-%
$C_1, \ldots, C_k$,
אז לכל
$i \neq j$
מתקיים
$C_i \cap C_j = \emptyset$
וגם
$\bigcup_{i=1}^k C_i = V$.
גרף הרק"ה של $G$ יסומן ב-%
$G_{scc} = (v_1, \ldots, v_k}, E_{scc})$
כאשר
$E_{scc} = \{(v_i, v_j) : \exsits (u, v) \in E, u \in C_i \land v \in C_j\}$.
ניתן לחשוב על גרף זה כגרף המתקבל על ידי כיווץ הרק"ה של $G$ וביטול קשתות מקבילות.

\textbf{דוגמה:}
\begin{center}
\begin{tikzpicture}[every node/.style={default node}, ->, x=1.5cm, y=1.5cm]

\begin{scope}[xshift=-7cm]
% NODES
\foreach[count=\i] \x \y in {
	0/0,0/1
	,1/0,1/1
	,2/0,2/1
	,3/0,3/1
}{
	\node(\i) at(\x,\y) {\i};
}

% EDGES
\foreach \u \v in{%
	1/2,1/3%
	,2/4%
	,4/1,4/3,4/6%
	,5/7%
	,6/5%
}{
	\draw[] (\u) -- (\v);
}

\foreach \u \v in{%
	3/5%
	,5/3%
	,6/8%
	,8/6%
}{
	\draw[] (\u) to[bend right] (\v);
}
\end{scope}

\begin{scope}[-, dashed, very thick]

\draw[orange!50]
(1.south west)	to[out=135, in=225]
(2.north west)	to[out=45, in=135]
(4.north east)	to[out=-45,in=-45,looseness=.7]
(1.south west)
;

\draw[blue!50]
(3.west)	to[out=90, in=90]
(5.east)	to[out=270, in=270]
(3.west)
;

\draw[green!50]
(6.west)	to[out=90, in=90]
(8.east)	to[out=270, in=270]
(6.west)
;

\draw[red!50, looseness=2]
(7.west)	to[out=90, in=90]
(7.east)	to[out=270, in=270]
(7.west)
;

\end{scope}

% NODES
\node(1)[fill=orange] at(0,1) {};
\node(2)[fill=blue] at(1,0) {};
\node(3)[fill=green] at(2,1) {};
\node(4)[fill=red] at(3,0) {};

% EDGES
\foreach \u \v in{%
	1/2,1/3%
	,2/4%
	,3/2%
}{
	\draw[] (\u) -- (\v);
}





\end{tikzpicture}
\end{center}

\begin{observation}
גרף הרק"ה היטב חסר מעגלים.
\end{observation}
\begin{proof}
אם קיים מעגל אז קיבלנו סתירה להגדרה של הגרף.

\end{proof}

\textbf{שאלה:}
איך נראה גרף הרק"ה של רשת כבישים? כיצד נראה גרף הרק"ה של ויקיפדיה? של דפי האינטרנט?

\textbf{מטרה:}
בהינתן גרף מכוון נרצה למצוא את גרף הרק"ה שלו.
פלט האלגוריתם צריך להיות מיפוי של כל צומת לרכיב קשיר היטב שמכיל אותה (מספר בין 1 ל-k).

נשים לב שבהינתן מיפוי כנ"ל ניתן לבנות את גרף הרק"ה על ידי מעבר בודד על קשתות הגרף המקורי.



עבור קבוצת צמתים
$C \subseteq V$
נרחיב את מושג הסיום מאלגוריתם 
\textenglish{DFS}
כך:
$\beta(C) = \max_{v \in C}\beta(v)$.
כלומר זמן הסיום המאוחר ביותר של צומת בקבוצה.

