בהינתן גרף מכוון, 
$G = (V, E)$
נגדיר את היחס הבא:
$R_\mathcal{C} = \{(u,v) : u \rightsquigarrow v \land v \rightsquigarrow u\}$,
כלומר צמתים $u$ ו-$v$ ביחס אם קיים מסלול מ-$u$ ל-$v$ וקיים מסלול מ-$v$ ל-$u$.

\textbf{תזכורת:}
יחס, $R$, הוא יחס שקילות מעל
$A \times A$
אם:
\begin{enumerate}
\item
רפלקסיביות - לכל
$a \in A$
מתקיים ש-%
$(a,a) \in R$.
\item
סימטריות - 
$(a,b) \in R \implies (b,a) \in R$
\item
טרנזיטיביות - 
$(a,b) \in R, (b,c) \in R \implies (a,c) \in R$
\end{enumerate}

נשים לב ש-%
$R_\mathcal{C}$
הוא יחס שקילות ולכן הוא מגדיר מחלקות שקילות. 
למחלקות שקילות אלו נקרא קבוצת הרכיבים קשירים היטב (רק"ה) של $G$.

\textbf{דוגמה:}
\begin{center}
\begin{tikzpicture}[every node/.style={default node}, ->, x=1.5cm, y=1.5cm]

% NODES
\foreach[count=\i] \x \y in {
	0/0,0/1
	,1/0,1/1
	,2/0,2/1
	,3/0,3/1
}{
	\node(\i) at(\x,\y) {\i};
}

% EDGES
\foreach \u \v in{%
	1/2,1/3%
	,2/4%
	,4/1,4/3,4/6%
	,5/7%
	,6/5%
}{
	\draw[] (\u) -- (\v);
}

\foreach \u \v in{%
	3/5%
	,5/3%
	,6/8%
	,8/6%
}{
	\draw[] (\u) to[bend right] (\v);
}

\begin{scope}[xshift=7cm]
% NODES
\foreach[count=\i] \x \y in {
	0/0,0/1
	,1/0,1/1
	,2/0,2/1
	,3/0,3/1
}{
	\node(\i) at(\x,\y) {\i};
}

% EDGES
\foreach \u \v in{%
	1/2,1/3%
	,2/4%
	,4/1,4/3,4/6%
	,5/7%
	,6/5%
}{
	\draw[] (\u) -- (\v);
}

\foreach \u \v in{%
	3/5%
	,5/3%
	,6/8%
	,8/6%
}{
	\draw[] (\u) to[bend right] (\v);
}
\end{scope}

\begin{scope}[-, dashed, very thick]

\draw[orange!50]
(1.south west)	to[out=135, in=225]
(2.north west)	to[out=45, in=135]
(4.north east)	to[out=-45,in=-45,looseness=.7]
(1.south west)
;

\draw[blue!50]
(3.west)	to[out=90, in=90]
(5.east)	to[out=270, in=270]
(3.west)
;

\draw[green!50]
(6.west)	to[out=90, in=90]
(8.east)	to[out=270, in=270]
(6.west)
;

\draw[red!50, looseness=2]
(7.west)	to[out=90, in=90]
(7.east)	to[out=270, in=270]
(7.west)
;

\end{scope}

\end{tikzpicture}
\end{center}

\begin{claim}
אם שני צמתים, $u$ ו-$v$ שייכים לאותו הרק"ה ובנוסף מתקיים ש-%
$u \rightsquigarrow w \rightsquigarrow v$
אז גם $w$ שייך לאותו הרק"ה.
\end{claim}

\begin{proof}
מספיק להראות ש-%
$w \rightsquigarrow u$
וזה נכון כיוון ש-%
$w \rightsquigarrow v$
(נתון)
ו-%
$v \rightsquigarrow u$
(באותו הרק"ה).
\end{proof}

נסמן את קבוצת הרק"ה של גרף ב-%
$\mathcal{C} = \{C_1, \ldots, C_k\}$,
אז לכל
$i \neq j$
מתקיים
$C_i \cap C_j = \emptyset$
וגם
$\bigcup_{i=1}^k C_i = V$
(יחס שקילות).
גרף הרק"ה של $G$ יסומן ב-%
$G_{scc} = (\mathcal{C}, E_{scc})$
כאשר
$E_{scc} = \{(C_i, C_j) : \exists\;(u, v) \in E, u \in C_i \land v \in C_j\}$.
ניתן לחשוב על גרף זה כגרף המתקבל על ידי כיווץ הרק"ה של $G$ וביטול קשתות מקבילות.

\pagebreak
\textbf{דוגמה:}
\begin{center}
\begin{tikzpicture}[every node/.style={default node}, ->, x=1.5cm, y=1.5cm]

\begin{scope}[xshift=-7cm]
% NODES
\foreach[count=\i] \x \y in {
	0/0,0/1
	,1/0,1/1
	,2/0,2/1
	,3/0,3/1
}{
	\node(\i) at(\x,\y) {\i};
}

% EDGES
\foreach \u \v in{%
	1/2,1/3%
	,2/4%
	,4/1,4/3,4/6%
	,5/7%
	,6/5%
}{
	\draw[] (\u) -- (\v);
}

\foreach \u \v in{%
	3/5%
	,5/3%
	,6/8%
	,8/6%
}{
	\draw[] (\u) to[bend right] (\v);
}
\end{scope}

\begin{scope}[-, dashed, very thick]

\draw[orange!50]
(1.south west)	to[out=135, in=225]
(2.north west)	to[out=45, in=135]
(4.north east)	to[out=-45,in=-45,looseness=.7]
(1.south west)
;

\draw[blue!50]
(3.west)	to[out=90, in=90]
(5.east)	to[out=270, in=270]
(3.west)
;

\draw[green!50]
(6.west)	to[out=90, in=90]
(8.east)	to[out=270, in=270]
(6.west)
;

\draw[red!50, looseness=2]
(7.west)	to[out=90, in=90]
(7.east)	to[out=270, in=270]
(7.west)
;

\end{scope}

% NODES
\node(1)[fill=orange] at(0,1) {};
\node(2)[fill=blue] at(1,0) {};
\node(3)[fill=green] at(2,1) {};
\node(4)[fill=red] at(3,0) {};

% EDGES
\foreach \u \v in{%
	1/2,1/3%
	,2/4%
	,3/2%
}{
	\draw[] (\u) -- (\v);
}

\end{tikzpicture}
\end{center}

\begin{observation}
גרף הרק"ה חסר מעגלים.
\end{observation}
\begin{proof}
אם קיים מעגל אז קיבלנו סתירה להגדרה של הגרף.

\end{proof}

\textbf{שאלה:}
איך נראה גרף הרק"ה של רשת כבישים? כיצד נראה גרף הרק"ה של ויקיפדיה? של דפי האינטרנט?

\textbf{מטרה:}
בהינתן גרף מכוון נרצה למצוא את גרף הרק"ה שלו.
פלט האלגוריתם צריך להיות מיפוי של כל צומת לרכיב קשיר היטב שמכיל אותה (מספר בין 1 ל-k).

נשים לב שבהינתן מיפוי כנ"ל ניתן לבנות את גרף הרק"ה על ידי מעבר בודד על קשתות הגרף המקורי.



עבור קבוצת צמתים
$C \subseteq V$
נרחיב את זמן הסיום של אלגוריתם
\textenglish{DFS}
לקבוצת צמתים כך:
$f(C) = \max_{v \in C}f(v)$.
כלומר זמן הסיום המאוחר ביותר של צומת בקבוצה.


\begin{claim}
לכל גרף ולכל ריצת 
\textenglish{DFS}
הצומת הראשון שמתגלה בכל רק"ה הוא אב קדמון של כל שאר הצמתים באותו הרק"ה ביער ה-%
\textenglish{DFS}.
\end{claim}

\begin{proof}
נסתכל על רק"ה, $C$, ועל הצומת הראשון שמתגלה בו, $v$,
מכיוון ש-$C$ רק"ה אז בזמן גילוי $v$ קיים מסלול לבן מ-$v$ לכל צומת אחר ברק"ה וההוכחה
נובעת ישירות ממשפט המסלול הלבן.
\end{proof}
הטענה הבאה מתייחסת לגרף רק"ה.
\begin{claim}
אם
$(C_i,C_j) \in E_{scc}$
אז לכל ריצת 
\textenglish{DFS}
על הגרף המקורי $G$ מתקיים ש-%
$f(C_i) > f(C_j)$.
\end{claim}

\begin{proof}
נפריד לשני מקרים:

\textbf{מקרה ראשון:}
מבקרים ב-%
$C_i$
לפני שמבקרים ב-%
$C_j$.
אז קיים מסלול לבן מהצומת הראשון שמתגלה ב-%
$C_i$
לכל שאר הצמתים ב-%
$C_i$
וגם ב-%
$C_j$
ולפי משפט המסלול הלבן צומת זה יהיה אב קדמון של כל הצמתים הנ"ל ולכן יהיה עם זמן סיום המאוחר 
ביותר מבניהם.

\textbf{מקרה שני:}
מבקרים בצומת מ-%
$C_j$
לפני שמבקרים בצומת מ-%
$C_i$.
אז מכיוון שגרף הרק"ה חסר מעגלים אין מסלול מצומת ב-%
$C_j$
לצומת ב-%
$C_i$.
ולכן זמן הסיום של הצומת הראשון שמתגלה ב-%
$C_j$
יהיה מוקדם יותר מזמן הגילוי (ולכן גם הסיום) של כל צומת ב-%
$C_i$.
מצד שני, בזמן גילוי הצומת הראשון ב-%
$C_j$
קיים מסלול לבן לכל שאר הצמתים ב-%
$C_j$
ולכן הצומת הראשון יהיה אב קדמון של כל הצמתים ב-%
$C_j$
וזמן הסיום שלו יהיה המאוחר ביותר מבניהם.
\end{proof}

\textbf{דיון:}
נניח שעבור גרף $G$ גרף הרק"ה,
$G_{scc}$,
נראה כך:
\begin{center}
\begin{tikzpicture}[every node/.style={default node}]

% NODES
\foreach[count=\i] \x \y in{
	0/0
	,1/1,1/-1
	,2/1
	,3/-1
}{
	\node(\i) at(\x,\y) {$C_{\i}$};
}

% EDGES
\foreach \u \v in{%
	1/2,1/3%
	,2/4%
	,3/2,3/5%
	,4/5%
}{
	\draw[->] (\u) -- (\v);
}

\end{tikzpicture}
\end{center}
כעת, נניח שהרצנו
\textenglish{DFS}
על $G$.
לאיזה רק"ה שייך הצומת עם זמן הסיום המקסימלי? 
מה יקרה אם נבחר בו בתור הצומת הראשון בהרצת
\textenglish{DFS}?


\begin{definition}[גרף משוחלף]
הגרף המשוחלף של גרף מכוון,
$G = (V, E)$,
הוא הגרף המתקבל על ידי הפיכת כיוון הקשתות, כלומר
$G^T = (V, E^T)$
כאשר
$E^T = \{(u, v) : (v, u) \in E\}$.
\end{definition}

\begin{claim}
$(G_{scc})^T = (\mathcal{C}, (E_{scc})^T) = (\mathcal{C}, (E^T)_{scc}) = (G^T)_{scc}$.
\end{claim}

\begin{proof}
נשים לב שהפיכת כיווני הקשתות לא משנה את היחס
$R_\mathcal{C}$.
בנוסף על פי הגדרה
$$
\begin{array}{lll}
(E_{scc})^T = 
&
\{(C_i, C_j) : (C_j, C_i) \in E_{scc}\} =
\\&
\{(C_i, C_j) : \exists\;(v, u) \in E,  u \in C_i \land v \in C_j\} =
\\&
\{(C_i, C_j) : \exists\;(u, v) \in E^T, u \in C_i \land v \in C_j\} =
&
(E^T)_{scc}
\end{array}
$$
.
\end{proof}

\textbf{דוגמה:}
\begin{center}
\begin{tikzpicture}[every node/.style={default node}, ->, x=1.5cm, y=1.5cm]

\begin{scope}[xshift=-7cm]
% NODES
\foreach[count=\i] \x \y in {
	0/0,0/1
	,1/0,1/1
	,2/0,2/1
	,3/0,3/1
}{
	\node(\i) at(\x,\y) {\i};
}

% EDGES
\foreach \u \v in{%
	1/2,1/3%
	,2/4%
	,4/1,4/3,4/6%
	,5/7%
	,6/5%
}{
	\draw[] (\u) -- (\v);
}

\foreach \u \v in{%
	3/5%
	,5/3%
	,6/8%
	,8/6%
}{
	\draw[] (\u) to[bend right] (\v);
}
\end{scope}

\begin{scope}[-, dashed, very thick]

\draw[orange!50]
(1.south west)	to[out=135, in=225]
(2.north west)	to[out=45, in=135]
(4.north east)	to[out=-45,in=-45,looseness=.7]
(1.south west)
;

\draw[blue!50]
(3.west)	to[out=90, in=90]
(5.east)	to[out=270, in=270]
(3.west)
;

\draw[green!50]
(6.west)	to[out=90, in=90]
(8.east)	to[out=270, in=270]
(6.west)
;

\draw[red!50, looseness=2]
(7.west)	to[out=90, in=90]
(7.east)	to[out=270, in=270]
(7.west)
;

\end{scope}

% NODES
\foreach[count=\i] \x \y in {
	0/0,0/1
	,1/0,1/1
	,2/0,2/1
	,3/0,3/1
}{
	\node(\i) at(\x,\y) {\i};
}

% EDGES
\foreach \u \v in{%
	1/2,1/3%
	,2/4%
	,4/1,4/3,4/6%
	,5/7%
	,6/5%
}{
	\draw[] (\v) -- (\u);
}

\foreach \u \v in{%
	3/5%
	,5/3%
	,6/8%
	,8/6%
}{
	\draw[] (\u) to[bend right] (\v);
}

\begin{scope}[-, dashed, very thick]

\draw[orange!50]
(1.south west)	to[out=135, in=225]
(2.north west)	to[out=45, in=135]
(4.north east)	to[out=-45,in=-45,looseness=.7]
(1.south west)
;

\draw[blue!50]
(3.west)	to[out=90, in=90]
(5.east)	to[out=270, in=270]
(3.west)
;

\draw[green!50]
(6.west)	to[out=90, in=90]
(8.east)	to[out=270, in=270]
(6.west)
;

\draw[red!50, looseness=2]
(7.west)	to[out=90, in=90]
(7.east)	to[out=270, in=270]
(7.west)
;

\end{scope}

\begin{scope}[xshift=-7cm, yshift=-4cm]
% NODES
\node(1)[fill=orange] at(0,1) {};
\node(2)[fill=blue] at(1,0) {};
\node(3)[fill=green] at(2,1) {};
\node(4)[fill=red] at(3,0) {};

% EDGES
\foreach \u \v in{%
	1/2,1/3%
	,2/4%
	,3/2%
}{
	\draw[] (\u) -- (\v);
}
\end{scope}

\begin{scope}[yshift=-4cm]
% NODES
\node(1)[fill=orange] at(0,1) {};
\node(2)[fill=blue] at(1,0) {};
\node(3)[fill=green] at(2,1) {};
\node(4)[fill=red] at(3,0) {};

% EDGES
\foreach \u \v in{%
	1/2,1/3%
	,2/4%
	,3/2%
}{
	\draw[] (\v) -- (\u);
}
\end{scope}

\end{tikzpicture}
\end{center}

