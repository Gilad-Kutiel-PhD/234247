בהינתן גרף מכוון ותת עץ (מושרש) שלו מסווגים את קשתות הגרף ל-4 סוגים:
\begin{enumerate}
\item
קשתות עץ
\item
קשתות קדמיות
\item
קשתות אחוריות
\item
קשתות חוצות
\end{enumerate}
\textbf{הערה:}
בגרף לא מכוון נתייחס לקשתות קדמיות וקשתות אחוריות כקשתות אחוריות.
\begin{center}
\begin{tikzpicture}[every node/.style={default node}]

% NODES
\node(s) at (0,0) {s};
\foreach[count=\i] \x \y in {
	2/1
	,1/2
	,2/-1
	,3/0
	,4/1
}{
	\node(\i) at(\x,\y) {\i};
}

% EDGES
\foreach \u \v in{%
	s/1%
	,s/2%
	,1/3%
	,1/4%
	,4/5%
}{
	\draw[] (\u) -- (\v);
}

% CROSS
\draw[blue, dashed, ->] (2) -- (1);
\draw[orange, dotted, ->] (s) -- (3);
\draw[brown, dash dot, ->] (5) -- (1);


\begin{scope}[xshift=6cm, y=5mm, yshift=1cm, every node/.style={label}]
\draw[blue, dashed, ->] (0,0) -- (1,0) node[right=5mm]{\texthebrew{קשת חוצה}};
\draw[orange, dotted, ->] (0,1) -- (1,1) node[right=5mm]{\texthebrew{קשת קדמית}};
\draw[brown, dash dot, ->] (0,2) -- (1,2) node[right=5mm]{\texthebrew{קשת אחורית}};
\end{scope}


\end{tikzpicture}
\end{center}

\begin{claim}
\label{claim:back}
בגרף לא מכוון ועץ שהוא פלט של DFS אין קשתות חוצות
\end{claim}
\begin{proof}
באמצעות למת המסלול הלבן
\end{proof}

כלומר, כל גרף לא מכוון ניתן לפרק (לעץ DFS וקבוצת קשתות אחוריות.
