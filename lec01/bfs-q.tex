ניתן לממש BFS על ידי תור באופן הבא:
\begin{enumerate}
\item
אתחול:
$U \leftarrow \{s\}, F \leftarrow \emptyset$, 
לכל 
$v \in V$
מציבים
$p(v) \leftarrow nil, d(v) \leftarrow \infty$,
$d(s) \leftarrow 0, Q \leftarrow (s)$
\item 
כל עוד התור לא ריק 
\begin{enumerate}
	\item
	$u \leftarrow Q.pop()$
\item
כל עוד ישנה קשת 
$uv$
שחוצה את $U$
($u \in U$)
		\begin{enumerate}
		\item
		$U \leftarrow U \cup \{v\}$,
		$F \leftarrow F \cup \{uv\}$,
		$p(v) \leftarrow u$
		\item
		$d(v) = d(u) + 1$
		\item
		$Q.push(v)$
		\end{enumerate}
	\end{enumerate}
\end{enumerate}
נראה שזהו אכן מימוש של BFS.
\begin{definition}[צומת גבולי]
בהינתן גרף 
$G = (V,E)$
וחתך 
$U \subseteq V$
צומת 
$u \in U$
יקרא 
\emph{גבולי}
אם קיימת קשת $uv$ שחוצה את $U$.
\end{definition}

\begin{claim}
בכל שלב בריצת האלגוריתם התור מכיל את כל הצמתים הגבוליים
\end{claim}
\begin{proof}
באינדוקציה על צעד האלגוריתם 
\end{proof}

\begin{claim}
התור מונוטוני לא יורד בהתייחס לערכי 
$d$
\end{claim}
\begin{proof}
נוכיח טענה חזקה יותר באינדוקציה על צעד האלגוריתם: 
התור מונוטוני לא יורד וגם 
$|d(u) - d(v)| \leq 1$
לכל שני צמתים שבתור
\end{proof}

\begin{corollary}
זהו אכן מימוש של BFS
\end{corollary}

\textbf{סיבוכיות:}
נשים לב שבמימוש הנ"ל כל צומת נכנס ויוצא מהתור לכל היותר פעם אחת 
ובנוסף כל קשת נבדקת לכל היותר פעמיים ולכן זמן הריצה הוא 
$O(|V| + |E|)$

