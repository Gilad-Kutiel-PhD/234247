\tikzset{
	spell/.style={
		decorate
		,decoration={
			snake
			,segment length=2pt
			,amplitude=.5pt
		}
		,red!70
		,thin	
	}
}
רוצים לבצע תיקון של שגיאות איות, למשל:
\begin{center}
\begin{tikzpicture}

\node[rectangle, inner sep=0](word) at(0,0) {סוויבון} node[left=3mm](w);
\draw [spell] (word.south west) -- (word.south east);

\begin{scope}[xshift=-2cm, yshift=-2cm]
\foreach[count=\i] \w in {
	סביבון
	,סביון
	,סבון
}{
	\node(\i) at(0, \i) {\w};
	\draw[thin, dashed] (w) -- (\i);
}
\end{scope}


\end{tikzpicture}
\end{center}

כדי לדעת אילו תיקונים להציע רוצים למדוד את המרחק בין המחרוזת שהוקלדה לבין המילה המוצעת.

בהינתן א"ב 
$\Sigma$
נגדיר
$\Sigma' = \Sigma \cup \{\varepsilon\}$.
ונגדיר:
\begin{definition}[הרחבה]
מחרוזת
$s' \in \Sigma'^*$
היא הרחבה של
$s \in \Sigma^*$ 
אם לאחר מחיקת כל תווי ה-%
$\varepsilon$
מ-$s'$ מקבלים את $s$.
\end{definition}

בהינתן פונקציית משקל 
$w: \Sigma' \times \Sigma' \to \mathcal{R}$
המרחק בין שתי הרחבות בעלות אורך זהה, $n$, הוא:
$$\sum_{i = 1}^n w(s'_1[i], s'_2[i])$$

\begin{example}
\label{example:1}
\begin{tabular}{|c|c|c|c|c|c|c|}
\hline
ס & ו & ו & י & ב & ו & ן
\\\hline
ס & ב & $\varepsilon$ & י & ב & ו & ן
\\\hline
\end{tabular}
\end{example}

\begin{example}
\label{example:2}
\begin{tabular}{|c|c|c|c|c|c|c|c|}
\hline
ס 
& ו
& ו 
& י 
& ב
& $\varepsilon$  
& ו 
& ן
\\\hline
ס 
& $\varepsilon$ 
& $\varepsilon$ 
& $\varepsilon$ 
& ב 
& י 
& ו 
& ן
\\\hline
\end{tabular}
\end{example}

עבור פונקציית המשקל
$$
w(\alpha, \beta) = 
\begin{cases}
0	& \text{if}\; \alpha = \beta
\\
1	& \text{otherwise}
\end{cases}
$$
אז המרחק בדוגמה 
\ref{example:1}
הוא 2 ובדוגמה
\ref{example:2}
הוא 4.

\begin{definition}[מרחק]
המרחק בין שתי מחרוזות (לאו דווקא באורך זהה) מעל 
$\Sigma$
הוא המרחק המינימלי האפשרי בין כל שתי הרחבות שלהן מאורך זהה.
\end{definition}

\textbf{הערה:}
אם מניחים שפונקציית המשקל אי שלילית אז מספר ההרחבות הרלוונטיות הוא סופי.


בהינתן שתי מחרוזות רוצים לחשב את המרחק ביניהן.

\textbf{פתרון:}
נסמן 
$|s| = m$
ו-%
$|r| = n$
נגדיר
$\alpha(i, j)$
להיות המרחק בין 
$s[i] \ldots s[m]$
ל-%
$r[j] \ldots r[n]$
כלומר בין הסייפות של $s$ ו-$r$.

ונחשב
$$
\alpha(i, j) = \min
\begin{cases}
w(s[i], r[j]) + \alpha(i + 1, j + 1)
\\
w(\varepsilon, r[j]) + \alpha(i, j + 1)
\\
w(s[i], \varepsilon) + \alpha(i + 1, j)
\end{cases}
$$

כמו כן מתקיים ש:%
$$
\alpha(i, n + 1) = \sum_{l = i}^m w(s[i], \varepsilon)
$$
וגם:
$$
\alpha(m + 1, j) = \sum_{l = j}^m w(\varepsilon, s[j])
$$
