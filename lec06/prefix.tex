קוד $c$ יקרא חסר רישות אם לא קיימים
$a,b \in \Sigma$
כך ש-%
$c(a)$
רישא של
$c(b)$

קל לראות שקודים חסרי רישות ניתנים לפענוח וכן לפענוח מידי.
מעבר לכך המשפט הבא (ללא הוכחה) מראה שלמטרתנו מספיק להתמקד בקודים חסרי רישות.
\begin{theorem}
לכל קוד חד פענח $c$ קיים קוד חסר רישות $c'$ כך שלכל 
$a \in \Sigma$
מתקיים
$|c(a)| = |c'(a)|$.
\end{theorem}

\subsection*{קוד חסר רישות כעץ בינרי}
ניתן לייצג כל קוד חסר רישות כעץ בינרי, למשל את הקוד 
$c_1$
ניתן לייצג על ידי העץ הבא:

\begin{center}
\begin{tikzpicture}

\node(s) at(0,0) {};
\node(bcd) at(1,-1) {};
\node(b) at(0,-2) {B};
\node(cd) at(2,-2) {};
\node(c) at(1,-3) {C};
\node(d) at(3,-3) {D};

\begin{scope}
\node(a) at (-1,-1) {A};
\end{scope}

\draw (s) -- (a) node[label above]{1};
\draw (s) -- (bcd) node[label above] {0};
\draw (bcd) -- (b) node[label above] {1};
\draw (bcd) -- (cd) node[label above] {0};
\draw (cd) -- (c) node[label above] {1};
\draw (cd) -- (d) node[label above] {0};
\end{tikzpicture}
\end{center}
