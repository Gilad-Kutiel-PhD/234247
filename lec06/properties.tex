נבחן שלושה קידודים שונים לא"ב 
$\{A, B, C, D\}$

$$
c_1 = 
\begin{array}{|l|l|}
\hline
A &  1
\\
B &  01
\\
C &  001
\\
D &  000
\\
\hline
\end{array}
%
\quad
c_2 = 
\begin{array}{|l|l|}
\hline
A &  0
\\
B &  01
\\
C &  011
\\
D &  111
\\
\hline
\end{array}
%
\quad
c_3 = 
\begin{array}{|l|l|}
\hline
A &  1
\\
B &  01
\\
C &  011
\\
D &  111
\\
\hline
\end{array}
$$
באופן טבעי נדרוש שהקוד יהיה ניתן לפענוח (חד פענח), 
כלומר נרצה שההרחבה תהיה פונקציה חד חד ערכית.

\textbf{דוגמה:}
ניתן לפענח את 
$c_1$,
ו-%
$c_2$,
אבל לא את
$c_3$.

תכונה רצויה היא שנוכל לפענח כל תו ברגע שקראנו את המילה שמקודדת אותו (פענוח מידי).

\textbf{דוגמה:}
התכונה מתקיימת עבור 
$c_1$,
אבל לא מתקיימת עבור 
$c_2$,
ו-%
$c_3$.
