\begin{enumerate}
\item
אתחול:
$U \leftarrow \{s\}, F \leftarrow \emptyset$, 
לכל 
$v \in V$
מציבים
$p(v) \leftarrow nil, d(v) \leftarrow -1$,
$d(s) \leftarrow 0$,
$i \leftarrow 0$,
\colorbox{yellow}{$S \leftarrow (s)$}
\end{enumerate}

\item
כל עוד המחסנית לא ריקה
\begin{enumerate}
	\item
	\colorbox{yellow}{$u \leftarrow S.top()$}
	\item 
	אם קיימת קשת 
	$uv$
	שחוצה את $U$ 
	($u \in U$)
		\begin{enumerate}
		\item
		$U \leftarrow U \cup \{v\}, F \leftarrow F \cup \{uv\}$
		\item
		$p(v) \leftarrow u$
		\item
		$d(v) \leftarrow i$
		\item
		\colorbox{yellow}{$S.push(v)$}
		\end{enumerate}
	\item
	אחרת 
	\begin{enumerate}
		\item 
		\colorbox{yellow}{$u \leftarrow S.pop()$}
		\item 
		\colorbox{yellow}{$\beta(u) = i$}
	\end{enumerate}
	\item
	$i \leftarrow i + 1$
	\end{enumerate}
\end{enumerate}

\begin{claim}
בזמן ריצת האלגוריתם, כל הצמתים הגבוליים נמצאים במחסנית
\end{claim}
\begin{proof}
באינדוקציה על צעד האלגוריתם
\end{proof}
\begin{claim}
המחסנית מונוטונית עולה ביחס ל-%
$d$
\end{claim}
\begin{proof}
באינדוקציה על צעד האלגוריתם
\end{proof}

\begin{corollary}
זהו מימוש של DFS
\end{corollary}

\textbf{הערה:}
מכיוון שזמני הגילוי של הצמתים הם יחודיים (בשונה מהמרחקים שלהם למשל) אזי המימוש באמצעות מחסנית
שקול לכל מימוש אחר של DFS (הדבר אינו נכון לגבי מימוש של BFS באמצעות תור).
לכן, כל טענה לגבי המימוש באמצעות מחסנית תקפה עבור DFS באופן כללי.
