\begin{claim}
\label{claim:path}
בזמן ריצת DFS, הצמתים במחסנית, 
$s,\ldots,v$
הם המסלול ב-$T$ מ-$s$ ל-$v$
\end{claim}
\begin{proof}
באינדוקציה על צעד האלגוריתם
\end{proof}
\begin{corollary}
עבור שני צמתים $u$ ו-$v$, $v$ צאצא של $u$ ב-$T$ אם ורק אם $u$ נמצא במחסנית כאשר $v$ מוכנס 
אליה.
\end{corollary}
\begin{proof}
כיוון ראשון מיידי מטענה
\ref{claim:path}.
\\
כיוון שני גם מטענה 
\ref{claim:path}
כאשר האבחנה היא שבעץ, צומת $u$ הוא אב קדמון של $v$ אם ורק אם הוא נמצא על המסלול מ-$s$
ל-$v$
\end{proof}

\begin{definition}[צומת לבן]
בזמן ריצת האלגוריתם, נקרא לצמתים ב-$U$ 
\underline{שחורים}
ולשאר הצמתים 
\underline{לבנים}
\end{definition}

\begin{observation}
צומת יוצא מהמחסנית רק אחרי שכל שכניו שחורים.
\end{observation}

\begin{lemma}[המסלול הלבן]
צומת $v$ צאצא של צומת $u$ ב-$T$ אמ"מ כאשר $u$ 
מוכנס למחסנית קיים ממנו מסלול של צמתים לבנים לצומת $v$
\end{lemma}
\begin{proof}
כיוון 'אם' באינדוקציה על אורך המסלול.
הצומת הראשון במסלול שנכנס למחסנית מחלק את המסלול לשני מסלולים קצרים יותר.
פרט קטן אך חשוב אחד הצמתים במסלול אכן נכנס למחסנית.
\\
כיוון 'רק אם' נניח בשלילה ש-$v$ צאצא של $u$ אבל כל מסלול בניהם מכיל צומת שחור אחד לפחות
אז בזמן הכנסת $v$ למחסנית תוכן המחסנית מכיל את המסלול מ-$u$ ל-$v$ ולכן הכנסנו למחסנית 
צומת שחור - סתירה.
\end{proof}
